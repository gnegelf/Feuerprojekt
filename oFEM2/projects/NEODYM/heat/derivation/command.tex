% !TEX root = ./main.tex

%% The amssymb package provides various useful mathematical symbols
\usepackage{amssymb}
\usepackage{bbm}
\usepackage{amsmath, amsthm}
\usepackage{bbm}
\numberwithin{equation}{section}
\usepackage{subfigure}
%\usepackage[osf,sc]{mathpazo}
%\usepackage{microtype}
%\usepackage{ellipsis}
\usepackage{nicefrac}
\usepackage{graphicx}
\usepackage{relsize}
\usepackage{siunitx}

\usepackage{color}
\usepackage{tikz}
\usepackage{tikz-3dplot}
\usepackage{tikzscale}
\usepackage{pgfplots}
\usetikzlibrary{plotmarks}
\usetikzlibrary{calc}
\usetikzlibrary{decorations.markings}
 % and optionally (as of Pgfplots 1.3):
\pgfplotsset{compat=newest}
\pgfplotsset{plot coordinates/math parser=false}
\newlength\figureheight
\newlength\figurewidth
\usetikzlibrary{shapes,arrows,chains, 3d, calc}
 %%%<
\usepackage[export]{adjustbox}




%\usepackage{ntheorem}
%\usepackage[thmmarks, amsmath, standard]{ntheorem}
\usepackage{listings}
\usepackage{url}
\usepackage{booktabs}
\usepackage{multirow}
\numberwithin{figure}{section}




%\newtheorem{alg}{Algorithm}




\def\vec#1{\mathchoice{\mbox{\boldmath$\displaystyle#1$}}
{\mbox{\boldmath$\mathrmstyle#1$}}
{\mbox{\boldmath$\scriptstyle#1$}}
{\mbox{\boldmath$\scriptscriptstyle#1$}}}

%\def\squareforqed{\hbox{\rlap{$\sqcap$}$\sqcup$}}
%\def\qed{\ifmmode\squareforqed\else{\unskip\nobreak\hfil
%\penalty50\hskip1em\null\nobreak\hfil\squareforqed
%\parfillskip=0pt\finalhyphendemerits=0\endgraf}\fi}


\usepackage[]{hyperref}
\hypersetup{
%    bookmarks=true,         % show bookmarks bar?
%   unicode=false,          % non-Latin characters in Acrobatâ??s bookmarks
%    pdftoolbar=true,        % show Acrobatâ??s toolbar?
%    pdfmenubar=true,        % show Acrobatâ??s menu?
%    pdffitwindow=false,     % window fit to page when opened
%    pdfstartview={FitH},    % fits the width of the page to the window
%    pdftitle={My title},    % title
%    pdfsubject={Subject},   % subject of the document
%    pdfcreator={Creator},   % creator of the document
%    pdfproducer={Producer}, % producer of the document
%    pdfkeywords={keyword1} {key2} {key3}, % list of keywords
%    pdfnewwindow=true,      % links in new window
    colorlinks=true,       % false: boxed links; true: colored links
    linkcolor=red,          % color of internal links (change box color with linkbordercolor)
    citecolor=green,        % color of links to bibliography
    filecolor=magenta,      % color of file links
    urlcolor=cyan           % color of external links
}

\definecolor{hsured}{RGB}{197,0,66}
\definecolor{hsublue}{RGB}{0,48,71}
\definecolor{hsugreen}{RGB}{0,113,106}
\definecolor{hsugray}{RGB}{144, 132,118}
% theorem, definition, ...
%\newtheoremstyle{mythmstyle}
%	{1cm}
%	{3pt}
%	{\itshape}
%	{}
%	{\bfseries}
%	{}
%	{\newline}
%	{\thmname{#1}\thmnumber{ #2}\thmnote{ (#3)}}
%
%\theoremstyle{mythmstyle}
\newtheorem{thm}{Theorem}[section]
\newtheorem{rem}[thm]{Remark}
\newtheorem{problem}[thm]{Problem}

\newcommand{\from}{\,:\,}
\newcommand{\bmat}[1]{\mathbf{#1}}
\newcommand{\bvec}[1]{\boldsymbol{#1}}
\newcommand{\Sp}[2]{\left\langle\, #1 , #2 \,\right\rangle}
\newcommand{\norm}[1]{\left\| \, #1 \, \right\|}
\newcommand{\Matrix}[1]{\begin{bmatrix} #1 \end{bmatrix}}
\newcommand{\Vector}[1]{\begin{pmatrix} #1 \end{pmatrix}}
\def\idx{\operatorname{idx}}
\DeclareMathOperator{\divergence}{div}
\DeclareMathOperator{\gradient}{grad}
\DeclareMathOperator{\differential}{D}

\newcommand{\R}{\ensuremath{\mathbbm{R}}}
\newcommand{\N}{\ensuremath{\mathbbm{N}}}
\newcommand{\C}{\ensuremath{\mathbbm{C}}}

\newcommand{\bp}{\ensuremath{\mathbf{p}}}
\newcommand{\by}{\ensuremath{\mathbf{y}}}
\newcommand{\bz}{\ensuremath{\mathbf{z}}}
\newcommand{\bd}{\ensuremath{\mathbf{d}}}

\newcommand{\Exp}[1]{\ensuremath{\mbox{e}^{#1}}}

\newcommand{\FC}{\ensuremath{\mathcal{C}}}
\newcommand{\divv}{\ensuremath{\operatorname{div}}}

\newcommand{\tr}{\ensuremath{\operatorname{tr}}}

\newcommand{\du}{\ensuremath{\partial \bvec{u}}}
\newcommand{\dr}{\ensuremath{\partial r}}
\newcommand{\ddu}{\ensuremath{\partial^2 \bvec{u}}}
\newcommand{\ddr}{\ensuremath{\partial r^2}}
\newcommand{\dt}{\ensuremath{\partial t}}
\newcommand{\ddt}{\ensuremath{\partial t^2}}
\newcommand{\dph}{\ensuremath{\partial \varphi}}

\newcommand{\D}{\ensuremath{\, \mathrm{d}}}
\newcommand{\Dt}{\ensuremath{\Delta t}}

\newcommand{\Du}{\ensuremath{\Delta \bvec{u}}}

\newcommand{\eps}{\ensuremath{\varepsilon}}
\newcommand{\ee}{\ensuremath{\varepsilon_{e}}}
\newcommand{\er}{\ensuremath{\varepsilon_{rr}}}
\newcommand{\et}{\ensuremath{\varepsilon_{\mathrm{tan}}}}

\newcommand{\De}{\ensuremath{\Delta \varepsilon}}
\newcommand{\Dee}{\ensuremath{\Delta \varepsilon_{e}}}
\newcommand{\Der}{\ensuremath{\Delta \varepsilon_{rr}}}
\newcommand{\Det}{\ensuremath{\Delta \varepsilon_{\mathrm{tan}}}}

\newcommand{\sig}{\ensuremath{\sigma}}
\newcommand{\sir}{\ensuremath{\sigma_{rr}}}
\newcommand{\sit}{\ensuremath{\sigma_{\mathrm{tan}}}}

\newcommand{\oot}{\ensuremath{\left ( t \right )}}
\newcommand{\oodt}{\ensuremath{\left ( t + \Dt \right )}}

\newcommand{\CC}{C\nolinebreak[4]\hspace{-.05em}\raisebox{.4ex}{\relsize{-3}{\textbf{++}}}}

\lstset{language=Matlab,%
    %basicstyle=\color{red},
    frame =lines,
    breaklines=true,%
    keywordstyle=\color{hsugreen},%
    identifierstyle=\color{hsugreen},%
    stringstyle=\color{hsured},
    commentstyle=\color{hsublue},%
    showstringspaces=false,%without this there will be a symbol in the places where there is a space
    numbers=left,%
    numberstyle={\tiny \color{black}},% size of the numbers
    numbersep=9pt, % this defines how far the numbers are from the text
    emph=[1]{for,end,break},emphstyle=[1]\color{hsured}, %some words to emphasise
    %emph=[2]{word1,word2}, emphstyle=[2]{style},    
}
%\lstset{language=Matlab}

\makeatletter
\tikzdeclarecoordinatesystem{barycentric}{{%
  \pgf@xa=0pt% point
  \pgf@ya=0pt%
  \pgf@xb=0pt% sum
  \tikz@bary@dolist#1,=,%
  \pgfmathreciprocal@{\pgf@sys@tonumber\pgf@xb}%
  \global\pgf@x=\pgfmathresult\pgf@xa%
  \global\pgf@y=\pgfmathresult\pgf@ya}}
\makeatother

\makeatletter
\def\ps@pprintTitle{%
 \let\@oddhead\@empty
 \let\@evenhead\@empty
 \def\@oddfoot{}%
 \let\@evenfoot\@oddfoot}
\makeatother

